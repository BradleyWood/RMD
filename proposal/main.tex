\documentclass[10pt, oneside,english]{article}
\usepackage{geometry}
\geometry{a4paper}
\usepackage[spanish, es-noindentfirst]{babel}
\selectlanguage{spanish}
\usepackage[utf8]{inputenc}

\usepackage{graphicx}

\usepackage{amssymb}
\usepackage{authblk}


\title{Code Migration for a Load Balanced Distributed Computing Platform}
\author[1]{Bradley Wood}
\affil[ ]{University of Ontario Institute of Technology}
\renewcommand\Authands{, }
\date{}

\begin{document}
    \maketitle

    \section{Overview}\label{sec:overview}

    Systems like Java RMI (Remote Method Invocation) allow developers to distribute
    workload across a network.
    However, systems like RMI require extensive setup on the server side which does
    not facilitate code migration.
    If the client wants to execute load balanced jobs across many machines
    a simple migration process could shield the developer from a client-server
    architecture and simplify the project setup.
    Java RMI also has other design limitations that prevent the client from execution static
    methods because they cannot be defined in an interface.

    \section{Objectives}\label{sec:objectives}

    \begin{itemize}
        \item Simplify distributed workloads on the JVM with code migration
        \item Provide load balancing across several machines
        \item Record and catalogue job results and statistics
    \end{itemize}

\end{document}
