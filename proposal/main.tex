\documentclass[12pt, oneside,english]{article}
\usepackage{geometry}
\geometry{a4paper}
\usepackage{float}
\usepackage{listings}
\usepackage{color}
\usepackage[english]{babel}
\selectlanguage{english}
\usepackage[utf8]{inputenc}

\usepackage{graphicx}
\usepackage{subcaption}
\usepackage{amssymb}
\usepackage{authblk}


\title{A Load Balanced Distributed Computing Platform with Code Migration}
\author[1]{Bradley Wood}
\author[2]{Zachary Winn}
\author[3]{Brock Watling}
\author[4]{Daniel Messiha}
\affil[ ]{University of Ontario Institute of Technology}
\renewcommand\Authands{, }
\date{}

\definecolor{dkgreen}{rgb}{0,0.6,0}
\definecolor{gray}{rgb}{0.5,0.5,0.5}
\definecolor{mauve}{rgb}{0.58,0,0.82}

\lstdefinelanguage{Kotlin}{
comment=[l]{//},
commentstyle={\color{gray}\ttfamily},
emph={delegate, filter, first, firstOrNull, forEach, lazy, map, mapNotNull, println, return@},
emphstyle={\color{OrangeRed}},
identifierstyle=\color{black},
keywords={abstract, actual, async, callback, delegate, as, as?, break, by, class, companion, continue, data, do, dynamic, else, enum, expect, false, final, for, fun, get, if, import, in, interface, internal, is, null, object, override, package, private, public, return, set, super, suspend, this, throw, true, try, typealias, val, var, vararg, when, where, while},
keywordstyle={\color{NavyBlue}\bfseries},
morecomment=[s]{/*}{*/},
morestring=[b]",
morestring=[s]{"""*}{*"""},
ndkeywords={@Deprecated, @JvmField, @JvmName, @JvmOverloads, @JvmStatic, @JvmSynthetic, Array, Byte, Double, Float, Int, Integer, Iterable, Long, Runnable, Short, String},
ndkeywordstyle={\color{BurntOrange}\bfseries},
sensitive=true,
stringstyle={\color{ForestGreen}\ttfamily},
}

\lstset{frame=tb,
language=Kotlin,
aboveskip=3mm,
belowskip=3mm,
showstringspaces=false,
columns=flexible,
basicstyle={\small\ttfamily},
numbers=none,
numberstyle=\tiny\color{gray},
keywordstyle=\color{blue},
commentstyle=\color{dkgreen},
stringstyle=\color{mauve},
breaklines=true,
breakatwhitespace=true,
tabsize=3
}

\begin{document}
    \maketitle

    \section{Overview}\label{sec:overview}

    Today, it is a challenging task to build secured,
    simple and efficient distributed systems.
    Often times, researchers are tasked with solve highly
    parallelizable problems that are too tough to be solved
    by a single machine. Existing systems have high barriers
    of entry, which require a complicated setup process.
    In this project, we aim to improve upon RPC systems
    by facilitating load balancing with code migration
    in cluster computing applications.
    Some existing RPC systems may implement code migration,
    but they do not provide any load balancing capabilities
    and Java RMI has design limitations that prevent the
    client from execution static methods because they cannot
    be defined in the interface.
    Our goal is to eliminate the barriers of entry to
    solving large scale distributed problems and to provide
    clients with limited programming experience the opportunity
    to parallelize large tasks over a network.

    \section{Objectives}\label{sec:objectives}

    \begin{itemize}
        \item[--] To streamline distributed workloads on the JVM
        \item[--] To provide load balancing across several machines
        \item[--] To allow clients to donate computing resources
        \item[--] To provide synchronous and asynchronous job requests
    \end{itemize}

    \section{Use Cases}\label{sec:useCases}

    \subsection{Cluster Computing}

    Cluster computing is a single logic unit connected with multiple
    computers that work together to maximize computing power.
    Cluster computing is effective when dealing with very difficult
    problems where a single processor is not adequate. Reasearchers could
    use the proposed system to find new prime numbers as this is a
    task which needs a vast amount of computing power to accomplish.
    Cluster computing has many advantages that we would like to mention.
    For example, Cost efficiency, processing speed, improved network
    infrastructure and high availability of resources.

    \subsection{Resource Sharing}

    The system should allow users the ability to share their
    resources for use through the network. This will give the
    users more freedom and flexibility to access the functionalities
    of the system. This functionality will benefit researchers as
    it gives them the ability to do what they desire by retrieving
    some resources that they could not obtain individual.
    Supports who donate resources could be compensated with a reward
    and such a system could help to build a thriving community.
    Supporters will also provided with a security guarantee against
    any malicious activity because the job requests will be executed
    with a strict security policy.

    \section{Architecture}\label{sec:architecture}

    \subsection{Code Migration}\label{subsec:migration}

    Inorder to execute instructions on a remote server, we must perform a migration action
    upon the function's first invocation. To do this, we must form a dependency graph
    by traversing each class file to determine all dependencies that are referenced by the class.
    This action guarantees that the server will have all the necessary resources
    to execute the job.

    \begin{figure}[H]
        \centering
        \includegraphics[scale=0.5]{migration.png}
        \caption{The migration process}
    \end{figure}

    \subsection{Load Balancing}\label{subsec:balancing}

    The load balancer should act as a middle-man between the client and the server
    and the client should not know the difference between communicating with
    a load balancer or with a job server.


    \begin{figure}[H]
        \centering
        \includegraphics[scale=0.5]{balancing.png}
        \caption{Interaction between entities}
    \end{figure}

    To balance a set of tasks we choose one of the following techniques:

    \begin{itemize}
        \item[--] Round robin
        \item[--] Lowest CPU usage first
    \end{itemize}

    A round robin job scheduler would be the easiest scheme to implement, however
    it comes with a few pitfalls.
    For example, in a round robin system, computers with weaker performance could
    become over utilized where as powerful computers could become underutilized.
    The algorithm would also not be abel to distinguish from short or long jobs
    causing the efficiency of the computing network to be unpredictable.
    A scheduling algorithm that prioritizes job servers based on the CPU usage
    is more ideal.
    However, this system is likely to have some degree of inaccuracy due to network
    latency.

    \subsubsection{Distributed Transparency}

    In some cases, the system could fail due to a variety of reasons including
    network and system failure or even a lack of servers to process the job.
    In these events, failure and recovery should be hidden from the client
    by making use of the following protocols.

    \begin{figure}[H]
        \begin{subfigure}{.5\textwidth}
            \centering
            \includegraphics[scale=0.7]{failure.png}
            \caption{No Resources Protocol}
        \end{subfigure}
        \begin{subfigure}{.5\textwidth}
            \centering
            \includegraphics[scale=0.5]{disconnect.png}
            \caption{Connection Lost Protocol}
        \end{subfigure}
    \end{figure}

    In the event that a user's job throws an exception, the server will return
    the exception information including it's detail message and stack trace.
    The client would sanitize the stack trace and then throw the exception
    in an attempt to make it look like the instructions had always been
    executing locally.


    \subsection{Proposed Usage}\label{subsec:proposageUsage}
    In theory, the system should be viable for any JVM language, and not just java.
    We plan to officially support the Java and Kotlin programming languages.
    Since the main objective of this project is to simplify the implementation
    of distributed software over a network, we have provided a few proposed
    examples to demonstrate both synchronous and asynchronous job execution.

    Regardless of the language, code that is to be executed on a remote job
    server will be denoted with a method reference or lambda expression.
    This is an important implementation detail because this technique allows
    the developer to maintain static type safety.

    \subsubsection{Synchronous Java Example}

    In this example, the function 'factorial' executed with an input of 6
    in a blocking manor. A method reference is used to tell the system what
    code is to be executed.

    \begin{lstlisting}
        System.out.println("6! " + delegate(Function::factorial, 6));
    \end{lstlisting}

    \subsubsection{Asynchronous Java Example}

    In this code snippet, the main calculation is done asynchronously
    by the server and the client does not block while waiting for the reply.
    the client is then free to do other tasks while the server completes the job.

    \begin{lstlisting}
        delegate((a, b) -> a * b, 5, 9, n -> {
            System.out.println("5 * 9 = " + n); // the callback
        });
    \end{lstlisting}

    \subsubsection{Synchronous Kotlin Example}

    Since Kotlin has extremely expressive syntax, a function which accepts a
    lambda expression can be written as a block with a body. For example, code
    in the delegate block would be executed remotely in a blocking manor.

    \begin{lstlisting}
        val a = 100
        val b = 200

        val result = delegate {
            a * b // blocking execution on remote server
        }

        println("$a * $b = $result")
    \end{lstlisting}

    \subsubsection{Asynchronous Kotlin Example}

    To execute an asynchronous job, the programmer can define code within an
    'async' block and optionally provide a 'callback' block to aquire the result.

    \begin{lstlisting}
        val a = 100
        val b = 200

        async {
            a * b // execute asynchrnously on job server
        } callback {
            println("$a * $b = $it") // disply result locally
        }

    \end{lstlisting}

\end{document}
