\section{FEATURES}\label{sec:features}

The platform shines in the field cluster computing because of it's
usability features and support for multiple JVM languages.
In fact, RMD maintains static typing, supports both
synchronous and asynchronous jobs, is configurable, and provides
a simple Kotlin DSL to aid the cleanliness of development.

\subsection{Static Type Safety}\label{subsec:typeSafety}

The platform is able to main strict type static safety by using
both method references and lambda expressions to denote
the job that you wish to delegate.
This technique allows the compiler to infer the type
information through a functional interface.
A functional interface is an interface who's type
can be used as a method reference or lambda expression.
To delegate a job, the user makes a call to one of the overloaded
delegate functions.
These delegate functions each accept one of several functional
interfaces that define functions which accept from 0 to 3 parameters
and produce a result, as well as the inputs to these functions.
Since the parameter types of the functional interface and the parameter inputs
are generic, the types must match or else the compiler will report
and error.


\subsection{Synchronous and Asynchronous Jobs}\label{subsec:synchronousAndAsynchronousJobs}

Both synchronous and asynchronous jobs are supported by the client.
Asynchronous jobs allow the machine to perform useful work
while jobs are executing remotely.
The use of synchronous is also important, and is highly applicable
to divide-and-conquer algorithms.
Such problem could be divided into several sub-problems which
execute asynchronously on the job servers.
Finally, a synchronous job could be used to combine the sub problems
into a single result.


\subsection{Configurability}\label{subsec:configurability}

The implementation provides several configurability options to the users.
In the config.kson file, (kson is an extension of json) users can
define a list of external hosts which could be either Job Servers,
or Load Balancer's.
Due to the distributed transparency, both types of external entities
are acceptable because the client, job server, and load balancer do
not distinguish between each other.
By default, users can specify up to three ways to handle
a lack of external computing resources.
Users can choose between a retry protocol, an exception based model,
and finally local execution.
In an exception based model, the application will simply throw
an exception if there is no available Job Server
to process the request.

The Load Balancer module is customizable and can used with other
job scheduling schemes.
A new scheme can be easily implemented by inheriting the BalanceStrategy
class.
Any BalanceStrategy can be passed to the Load Balancer because it depends upon
the abstraction, and not any concrete class.

\subsection{Kotlin DSL}\label{subsec:kotlinDsl}

In an attempt to provide platform support to other JVM languages, we
implemented a Kotlin DSL (domain specific language).
A domain specific language in this context is an internal DSL, which
means that it is built on top of the Kotlin and JVM infrastructure
to provide domain specific enhancements ie., cluster computing.
This approach allows the user to define their jobs within blocks
that are specific to the cluster computing application.
For the Kotlin DSL, we provide three types of blocks:
delegate, async, and callback.
The delegate block is a block of code that represents a job
that is to be executed synchronously on the remote job server.
The last statement in the block is also the result returned to the
programmer.
The async and callback blocks are designed to go together because
a callback is required to retrieve the result from an asynchronous
job.
The result of the last statement in the async block is the input
to the callback.
The Kotlin DSL serves to enhance the features of the language and to
provide a cleaner way to define jobs, which does not depend upon
implementing any interface.
