\section{FAILURE PROTOCOLS}\label{sec:conclusions}

Failures are inevitable in a large scale cluster computing network.
Such failures are most likely to be caused by network errors, which
is why several failure protocols have been designed and implemented.
These failures can occur both before or after the successful
migration and while executing the jobs.
In general, failures are ignored and the job requests are passed
on to the next available job server.
However, this is not possible if no computing resources are
found on the network.
Handling this type of error can be configured
using a configuration file.
Currently, there are three supported protocols to handle
this type of failure.
In the first protocol, jobs are to be executed locally when
there are no external computing resources available.

\begin{figure}[H]
    \centering
    \includegraphics[scale=0.50]{failure.png}
    \caption{Local Execution on Failure}
\end{figure}

Secondly, such events can be handled with a wait and retry protocol.
If users decide to choose this approach, they will not be bogged down
by the local execution of the jobs, but they must accept that any network
issues may not resolve themselves and no error messages are displayed.

\begin{figure}[H]
    \centering
    \includegraphics[scale=0.70]{failure2.png}
    \caption{Wait and Retry}
\end{figure}

Finally, the last protocol is the error protocol.
If a lack of resources ever occurs, the client will
throw an exception indicating that no external resources
are available.
If the user chooses this scheme, they should be careful
to catch any exceptions that could be thrown by delegating
resources by using a try-catch block.
If uncaught, the JVM will halt the execution of the thread
that threw the exception.

Sometimes failure can occur after a job request has been issued.
In this event, it is preferred to cut losses and pass on the request
to the next available job server.
This design decision was chosen because of the uncertainty surrounding
the other nodes in the network.
A cache based system could have been implemented, but given that the duration
of each job execution should be relatively small, it is unlikely that the issue
would be resolved within that time frame.
This could require the client to attempt to reconnect to a server to retrieve
the result with no guarantees of success, all while wasting time.

\begin{figure}[H]
    \centering
    \includegraphics[scale=0.70]{disconnect.png}
    \caption{Disconnect Protocol}
\end{figure}
