\section{IMPLEMENTATION}\label{sec:impl}

The project has been implemented on the Java Virtual Machine with a modular design allowing for
a separation of concerns.
In total, there are 5 modules including the load balancer, job server, the main client,
the Kotlin client (Kotlin DSL support), and the communications module.
Each module is responsible for one specific job, and they collectively work together.

\subsection{Load Balancer}\label{subsec:modules}

The load balancer has the important job of managing the use of resources
on the computing network.
This module depends upon the communication module to handle all network IO
between clients and job servers.
The load balancer accepts connections from clients and forwards their job
request based on resource availability.
It also also clients to self describe as a job server in order to allow for
resource donation.
To balance the computing load across several job servers, a round-robin
scheduling algorithm is provided as the default scheduling algorithm.
Advanced users can write and use their own scheduling algorithms via dependency
injection.
After a client initialized a connection, it must first send a migration request
to the load balancer before sending any job requests.
This protocol acts the same as direct communication between a client and server,
therefore the client does not need to distinguish between a job server or load
balancer.
The load balancer must verify that a job server has the required
code before forwarding a job request to the chosen server.
This is a problem because of the one-to-many relationship between the client
and job servers.
To solve this problem, the load balancer stores a map between
a JVM class name, and its bytecode instructions.
It also records a list of each class file that the job server has so it can
prevent migration of instructions that already exist on the job server.
The migration process is lazy, and only occurs when a job request
occurs to a server that has not seen the specific code before.