\section{INTRODUCTION}\label{sec:introduction}

Today, it is a challenging task to build secured, simple and efficient distributed systems. Often times, researchers are tasked with solving highly parallelizable problems that are too tough to be solved by a single machine. These machines just don’t have the amount of resources that are needed and some jobs may exceed what it is capable of. However in this situation and many other similar ones, if enough machines share their resources then these problems can be solved without the use of large spending. In order to utilize this, platforms use cluster computing which allows users with extra resources to donate some of their available resources to help other users with objectives that need them.

We provide load balancing across multiple machines. This is important because it will ensure that none of the machines in the system will be overutilized or underutilized. Other solutions might end up with one machine doing the work of several, so the result is a much slower system.

Clients are able to donate their computing resources to help complete incoming requests. This will allow our computing network to become even more powerful, and complete tasks with higher processing requirements in less time.

The system can handle both synchronous and asynchronous jobs, which means we can support clients that need results immediately, and clients that want to continue processing and receive a notification when the job is complete.