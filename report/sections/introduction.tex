\section{INTRODUCTION}\label{sec:introduction}


Today, it is a challenging task to build secured, simple and efficient distributed
systems. Often times, researchers are tasked with solving highly parallelizable problems
that are too tough to be solved by a single machine. Existing systems have
high barriers of entry, which require a complicated setup process. In this project has improved upon RPC systems by facilitating load balancing with code
migration in cluster computing applications. Some existing RPC systems may implement
code migration, but they do not provide any load balancing capabilities
and Java RMI has design limitations that prevent the client from executing static
methods because they cannot be defined in the interface. Our goal is to eliminate
the barriers of entry to solving large-scale distributed problems and to provide
clients with limited programming experience the opportunity to parallelize large
tasks over a network.

Our project has increased usability over other examples of RPC systems with code migration. It streamlines distributed workloads on the JVM so that the user only needs to make the request and specify any dependencies.

We provide load balancing across multiple machines. This is important because it will ensure that none of the machines in the system will be overutilized or underutilized. Other solutions might end up with one machine doing the work of several, so the result is a much slower system.

Clients are able to donate their computing resources to help complete incoming requests. This will allow our computing network to become even more powerful, and complete tasks with higher processing requirements in less time.

The system can handle both synchronous and asynchronous jobs, which means we can support clients that need results immediately, and clients that want to continue processing and receive a notification when the job is complete.