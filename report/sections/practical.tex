\section{PRACTICAL USE}\label{sec:practicalUse}

Although RMD is simple to setup and configure,
best practices should to followed to improve the
efficiency of your system.


\subsection{Project Setup}\label{subsec:projectSetup}

To setup your project, simply add the dependency to your
project.
If you are using Kotlin, your project should depend
on the rmd-kotlin module to take advantage of
the features provided by the Kotlin DSL.
If you are using Java (or other JVM languages), your project
should depend on the rmd-client module.

\begin{figure}[H]
\caption{Kotlin Coordinates}
    \begin{lstlisting}
ca.uoit.rmd:rmd-kotlin:$rmd-version
    \end{lstlisting}
\end{figure}

\begin{figure}[H]
    \caption{Java Coordinates}
    \begin{lstlisting}
ca.uoit.rmd:rmd:$rmd-version
    \end{lstlisting}
\end{figure}

Finally, the project should be configured by creating
a config.kson file to specify the location of any
external load balancer's and job servers. You may also specify
an error strategy, in the case that no external servers are found.

\begin{figure}[H]
\caption{config.kson}
    \begin{lstlisting}
config {
    hosts: {
        "35.243.160.240"
        "localhost"
    }
}
    \end{lstlisting}
\end{figure}

\subsection{Writing Jobs}\label{subsec:writingJobs}
Before you begin implementing your jobs, check for
existing algorithms.
It is often the case that you do not need to define
your jobs as existing algorithms can be used via a method reference.

When writing your own custom jobs, try to limit the number of
dependencies used by your job.
More dependencies will affect the initialization overhead cost
as this will require the client to detect, locate, and migrate
more code at runtime.


\subsection{Setting up Servers}\label{subsec:settingUpServers}

