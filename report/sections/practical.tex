\section{PRACTICAL USE}\label{sec:practicalUse}

Although RMD is simple to setup and configure,
best practices should be followed to improve the
efficiency of your system.


\subsection{Project Setup}\label{subsec:projectSetup}

To setup your project, simply add the dependency to your
project.
If you are using Kotlin, your project should depend
on the rmd-kotlin module to take advantage of
the features provided by the Kotlin DSL.
If you are using Java (or other JVM languages), your project
should depend on the rmd-client module.

\begin{figure}[H]
\caption{Kotlin Coordinates}
    \begin{lstlisting}
ca.uoit.rmd:rmd-kotlin:$rmd-version
    \end{lstlisting}
\end{figure}

\begin{figure}[H]
    \caption{Java Coordinates}
    \begin{lstlisting}
ca.uoit.rmd:rmd:$rmd-version
    \end{lstlisting}
\end{figure}

Finally, the project should be configured by creating
a config.kson file to specify the location of any
external load balancers and job servers. You may also specify
an error strategy, in the case that no external servers are found.

\begin{figure}[H]
\caption{config.kson}
    \begin{lstlisting}
config {
    hosts: {
        "35.243.160.240"
        "localhost"
    }
}
    \end{lstlisting}
\end{figure}

\subsection{Writing Jobs}\label{subsec:writingJobs}
Before you begin implementing your jobs, check for
existing algorithms.
It is often the case that you do not need to define
your jobs as existing algorithms can be used via a method reference.

When writing your own custom jobs, try to limit the number of
dependencies used by your job.
More dependencies will affect the initialization overhead cost
as this will require the client to detect, locate, and migrate
more code at runtime.
Jobs should be designed to take longer than the network
delay however, if they are too long, a server side failure
could result in a large delay to computation time.
It is suggested to write jobs that execute in 1-10 seconds.

If you plan to execute more than one asynchronous job at the
same time, pay attention to common concurrency pitfalls.
Your callback methods should synchronize the modification
of global variables, either by using a synchronized block or
atomic references.
Failure to do so may result in undesirable behaviour.
For example, consider a program that executes 1000 asynchronous
jobs and their callback function increments a global counter.
It is extremely likely that your counter may not be equal to 1000
after completion.

Finally, pay attention to the context in which your job executes.
Your job should not attempt to modify the state of any variable
outside the scope of the job.
Instead, a result should always be returned by the job.
You may however accept the state of local and global variables
as input, so long as they are constant and immutable.
Modification of the state of any object or reference outside the job
will not transfer back to the master program.

\subsection{Setting up Servers}\label{subsec:settingUpServers}

Both the load balancer and job server modules act as standalone
applications.
You simply need to download the build artifacts and write
a very simple configuration file.


\subsubsection{Load Balancer}

The configuration file of the load balancer should point
to the known job server in the network.
By making use of the load balancer, clients do not
need to know about all the job servers on the network.
This allows administrators to easily scale up/down the
size of the cluster without taking the clients offline for
maintenance.
The geographic location of the load balancer should
be taken into account because the added networking delay
may not be worth the advantages of providing an external
load balancer.

For advanced usage, the load balancer's behaviour can
be programmatically modified to exhibit behaviour that is
more beneficial to your use case.
For example, you may want to implement your own scheduling
algorithm that prioritizes lowest CPU usage, or lowest ping
first.

\subsubsection{Job Server}

The Job Server should work out of the box, without any configuration.
